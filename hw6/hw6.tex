\documentclass{article}
\usepackage[margin=1in]{geometry}
\usepackage{amsmath, amssymb}
\usepackage{enumerate}
\usepackage{fancyhdr}
\usepackage{titlesec}
\usepackage{verbatim}
\usepackage{/usr/share/R/share/texmf/tex/latex/Sweave}


\pagestyle{fancy}
\fancyhead[L]{STAT 579 HW6}
\fancyhead[R]{Xin Yin}
\titleformat{\section}{\em\Large} {$\dagger$ \thesection .}{10pt}{}

\begin{document}
\section{Exercises 2/3/4}
To calculate $h(x, n) = \sum_{i=0}^n x^i = 1 + x + \dots + x^n$ explicitly with a \verb=for= loop, we can first start with an obvious approach.
\begin{Schunk}
\begin{Sinput}
> h.naive <- function(x, n) {
+     x.sum <- 0
+     for (i in 0:n) {
+         x.sum <- x.sum + x^i
+     }
+     return(x.sum)
+ }
\end{Sinput}
\end{Schunk}
Let's check if this will work out,
\begin{Schunk}
\begin{Sinput}
> c(h.naive(0.3, 55), h.naive(6.6, 8))
\end{Sinput}
\begin{Soutput}
[1] 1.428571e+00 4.243336e+06
\end{Soutput}
\end{Schunk}
Well, it turns out this is not fun at all. It turns out that this sum can be computed in a recursive manner, namely, 
\[
h(x, n) = h(x, n-1)\cdot x + 1
\]
Notice that because $h(x, n-1)$ is a constant, this is a linear transform that we can write down in a matrix form as,
\[
h(x, n) = 
    \begin{pmatrix}
        x & 1 \\
        0 & 1
    \end{pmatrix}
    \begin{pmatrix}
        h(x, n-1) \\
        1
    \end{pmatrix}
\]

So,
\[
h(x, n) = 
    \begin{pmatrix}
        x & 1 \\
        0 & 1
    \end{pmatrix}^{(n)}
    \begin{pmatrix}
        1 \\
        1
    \end{pmatrix}.
\]

Unfortunately, \verb=R= has no built-in function to calculate a matrix to the power of $n$. On the other side of the coin, we can then implement this power of a matrix using \verb=for= loop in comply with the requirement of exercise 2.
\begin{Schunk}
\begin{Sinput}
> h.matrix <- function(x, n) {
+     ltf.m <- m <- matrix(c(x, 0, 1, 1), ncol = 2)
+     v <- c(1, 1)
+     for (i in 2:n) {
+         m <- ltf.m %*% m
+     }
+     return(m %*% v)
+ }
> c(h.matrix(0.3, 55)[1], h.matrix(6.6, 8)[1])
\end{Sinput}
\begin{Soutput}
[1] 1.428571e+00 4.243336e+06
\end{Soutput}
\end{Schunk}

Now we jump to exercise 4. First we rewrite our \verb=naive= version of $h(x, n)$ using \verb=while= loop.
\begin{Schunk}
\begin{Sinput}
> h.naive.w <- function(x, n) {
+     x.sum <- 0
+     while (n >= 0) {
+         x.sum <- x.sum + x^n
+         n <- n - 1
+     }
+     return(x.sum)
+ }
> c(h.naive.w(0.3, 55), h.naive.w(6.6, 8))
\end{Sinput}
\begin{Soutput}
[1] 1.428571e+00 4.243336e+06
\end{Soutput}
\end{Schunk}

Obviously we can rewrite the \verb=h.matrix= using \verb=while= loop as well, but why bother?

Of course, they are all kinds of buzz about now inefficient a \verb=for= loop is in \verb=R=. So, let's consider other implementations of $h(x, n)$, and compare the speed of all versions of $h(n, x)$ later.
First we can use the almighty \verb=apply= family to replace the notorious \verb=for= loop,
\begin{Schunk}
\begin{Sinput}
> h.apply <- function(x, n) return(1 + sum(sapply(1:n, function(i) x^i)))
> c(h.apply(0.3, 55), h.apply(6.6, 8))
\end{Sinput}
\begin{Soutput}
[1] 1.428571e+00 4.243336e+06
\end{Soutput}
\end{Schunk}

We can also improve the speed of our sluggish implementation of power of a matrix using some well optimizied package,
\begin{Schunk}
\begin{Sinput}
> library(expm)
> h.matrix.alt <- function(x, n) {
+     ltf.m <- matrix(c(x, 0, 1, 1), ncol = 2)
+     v <- c(1, 1)
+     return(ltf.m %^% n %*% v)
+ }
> c(h.matrix.alt(0.3, 55)[1], h.matrix.alt(6.6, 8)[1])
\end{Sinput}
\begin{Soutput}
[1] 1.428571e+00 4.243336e+06
\end{Soutput}
\end{Schunk}

Let's do some benchmarking at this point,
\begin{Schunk}
\begin{Sinput}
> system.time(replicate(10000, h.naive(0.3, 55)))
\end{Sinput}
\begin{Soutput}
   user  system elapsed 
  1.584   0.024   1.624 
\end{Soutput}
\begin{Sinput}
> system.time(replicate(10000, h.naive.w(0.3, 55)))
\end{Sinput}
\begin{Soutput}
   user  system elapsed 
  2.564   0.004   2.600 
\end{Soutput}
\begin{Sinput}
> system.time(replicate(10000, h.apply(0.3, 55)))
\end{Sinput}
\begin{Soutput}
   user  system elapsed 
  3.228   0.008   3.283 
\end{Soutput}
\begin{Sinput}
> system.time(replicate(10000, h.matrix(0.3, 55)))
\end{Sinput}
\begin{Soutput}
   user  system elapsed 
  1.376   0.004   1.384 
\end{Soutput}
\begin{Sinput}
> system.time(replicate(10000, h.matrix.alt(0.3, 55)))
\end{Sinput}
\begin{Soutput}
   user  system elapsed 
  0.316   0.000   0.315 
\end{Soutput}
\end{Schunk}

So, \verb=for= loop is, yes, slow. \verb=while= is even more horrible in speed. Interestingly, using \verb=sapply= to calculate $h(x,n)$ is super inefficient because of the extra overhead of function calls (well,function calls are expensive). The matrix calculation with \verb=for= loop is even a little bit faster than the \verb=naive= one. So, is it because the matrix calculation is more efficient than loop, or is it because to compute $x^i$ at every iteration of $i$ is computationally costly? 

To test this, we can rewrite the naive for loop version in order to avoid recomputing $x^i$ at each given $i$.

\begin{Schunk}
\begin{Sinput}
> h.naive.cum <- function(x, n) {
+     x.sum <- x.n <- 1
+     for (i in 1:n) {
+         x.n <- x.n * x
+         x.sum <- x.sum + x.n
+     }
+     return(x.sum)
+ }
> c(h.naive.cum(0.3, 55)[1], h.naive.cum(6.6, 8)[1])
\end{Sinput}
\begin{Soutput}
[1] 1.428571e+00 4.243336e+06
\end{Soutput}
\begin{Sinput}
> system.time(replicate(10000, h.naive.cum(0.3, 55)))
\end{Sinput}
\begin{Soutput}
   user  system elapsed 
  1.528   0.012   1.553 
\end{Soutput}
\end{Schunk}

As we can see, the improvement is negligible, if any. So, yes, evil \verb=for= loop is sloooooow. 

Finally, the matrix implementation using the \verb=expm= package is pretty efficient because we remove all the code that can be laggy in \verb=R=. But is this the most elegant and most efficient approach? Let's try something else, that without any \verb=for= loop and avoid repeatative function calls,
\begin{Schunk}
\begin{Sinput}
> h.cumprod <- function(x, n) 1 + sum(cumprod(rep(x, n)))
> c(h.cumprod(0.3, 55), h.cumprod(6.6, 8))
\end{Sinput}
\begin{Soutput}
[1] 1.428571e+00 4.243336e+06
\end{Soutput}
\begin{Sinput}
> system.time(replicate(10000, h.cumprod(0.3, 55)))
\end{Sinput}
\begin{Soutput}
   user  system elapsed 
  0.136   0.004   0.140 
\end{Soutput}
\end{Schunk}
Yes, \verb=R= does replication fast, calculates sum fast, and the \verb=cumprod= turns out to be very efficient.

If you turn out to be a functional programming kid, you may also write this
snippet of code to do the same job. So, no \verb=for=, no \verb=while=, looks clean. What's the performance then?
\begin{Schunk}
\begin{Sinput}
> h.fp <- function(x, n) 1 + Reduce("+", (Reduce("*", rep(x, n), 
+     accumulate = T)))
> c(h.fp(0.3, 55), h.fp(6.6, 8))
\end{Sinput}
\begin{Soutput}
[1] 1.428571e+00 4.243336e+06
\end{Soutput}
\begin{Sinput}
> system.time(replicate(10000, h.fp(0.3, 55)))
\end{Sinput}
\begin{Soutput}
   user  system elapsed 
  8.692   0.028   8.840 
\end{Soutput}
\end{Schunk}
If this suggests anything, it is simply that always think in R when working with R.

\section{Exercise 10}
Since this exercise is dumb and boring, let's do it fast,
\begin{Schunk}
\begin{Sinput}
> v.min <- function(x) {
+     x.min <- x[1]
+     for (i in 2:length(x)) {
+         if (x[i] < x.min) {
+             x.min <- x[i]
+         }
+     }
+     return(x.min)
+ }
\end{Sinput}
\end{Schunk}
It may not be robust to some weird \verb=x= that one may feed this function. But who cares? 

Test drive here,
\begin{Schunk}
\begin{Sinput}
> (xr <- ceiling(runif(20) * 100))
\end{Sinput}
\begin{Soutput}
 [1] 69 50 47 63 45 30 59 18 55 61 97 14 43 41 33 80 66 13 66 96
\end{Soutput}
\begin{Sinput}
> c(v.min(xr), min(xr))
\end{Sinput}
\begin{Soutput}
[1] 13 13
\end{Soutput}
\end{Schunk}

\section{Exercise 11}
This function is exactly the ``merge'' step in a merge sort algorithm. 
\begin{Schunk}
\begin{Sinput}
> merge <- function(x, y) {
+     lx <- length(x)
+     ly <- length(y)
+     lm <- lx + ly
+     m <- rep(0, lm)
+     i <- j <- 1
+     for (k in 1:length(m)) {
+         if (x[i] < y[j]) {
+             m[k] <- x[i]
+             i <- i + 1
+         }
+         else {
+             m[k] <- y[j]
+             j <- j + 1
+         }
+         if (i > lx) {
+             m[(k + 1):lm] <- y[j:ly]
+             break
+         }
+         if (j > ly) {
+             m[(k + 1):lm] <- x[i:lx]
+             break
+         }
+     }
+     return(m)
+ }
\end{Sinput}
\end{Schunk}
Let's try this out,
\begin{Schunk}
\begin{Sinput}
> x <- sort(rpois(8, lambda = 9))
> y <- sort(rpois(12, lambda = 6))
> sort(c(x, y))
\end{Sinput}
\begin{Soutput}
 [1]  3  4  4  5  5  6  6  7  7  7  8  8  8  9  9  9 10 11 11 12
\end{Soutput}
\begin{Sinput}
> merge(x, y)
\end{Sinput}
\begin{Soutput}
 [1]  3  4  4  5  5  6  6  7  7  7  8  8  8  9  9  9 10 11 11 12
\end{Soutput}
\begin{Sinput}
> all(merge(x, y) == sort(c(x, y)))
\end{Sinput}
\begin{Soutput}
[1] TRUE
\end{Soutput}
\end{Schunk}

\section{Exercise 12}
We can implement a function named \verb=craps= to simulate this (horrible) game for any arbitrary time. 
\begin{Schunk}
\begin{Sinput}
> craps <- function(n) {
+     results <- matrix(rep(0, 2 * n), nrow = 2)
+     dice.sum <- replicate(n, sum(ceiling(6 * runif(2))))
+     win.idx <- which(dice.sum %in% c(7, 11))
+     results[, win.idx] <- c(1, 1)
+     remain <- (1:n)[-win.idx]
+     first.round <- dice.sum[-win.idx]
+     round <- 2
+     while (length(remain)) {
+         dice.sum <- replicate(length(remain), sum(ceiling(6 * 
+             runif(2))))
+         win.idx <- dice.sum == first.round
+         results[, remain[win.idx]] <- c(1, round)
+         lose.idx <- dice.sum %in% c(7, 11)
+         results[, remain[lose.idx]] <- c(0, round)
+         remain <- remain[!(win.idx | lose.idx)]
+         first.round <- first.round[!(win.idx | lose.idx)]
+         round <- round + 1
+     }
+     return(results)
+ }
\end{Sinput}
\end{Schunk}

So now let's simulate playing this game for 10000 times.
\begin{Schunk}
\begin{Sinput}
> sim.craps <- data.frame(t(craps(10000)))
> names(sim.craps) <- c("outcome", "round")
> sim.craps[, "outcome"] <- factor(sim.craps[, "outcome"], levels = c(0, 
+     1), labels = c("L", "W"))
\end{Sinput}
\end{Schunk}
And how many times do we win?
\begin{Schunk}
\begin{Sinput}
> table(sim.craps[, "outcome"])
\end{Sinput}
\begin{Soutput}
   L    W 
5318 4682 
\end{Soutput}
\end{Schunk}
It slighly more likely to lose the the game. And we can also plot the histogram of how many rounds we need to either claim a win or lost.
\begin{Schunk}
\begin{Sinput}
> hist(sim.craps[, "round"], main = "Histogram of rounds of gameplay")
\end{Sinput}
\end{Schunk}

\begin{center}
\includegraphics[width=0.5\textwidth]{hw6-018}
\end{center}

\section{Exercise 13}
Let's plot this mosquito coil, shall we?
\begin{Schunk}
\begin{Sinput}
> t <- seq(0, 10, length.out = 500)
> phi <- 2 * pi * t
> r <- sqrt(t)
> plot(r * cos(phi), r * sin(phi), typ = "l", xlab = expression("x(t)"), 
+     ylab = expression("y(t)"))
\end{Sinput}
\end{Schunk}

\begin{center}
\includegraphics[width=0.5\textwidth]{hw6-019}
\end{center}
\end{document}
