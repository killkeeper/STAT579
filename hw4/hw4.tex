\documentclass{article}
\usepackage[margin=1in]{geometry}
\usepackage{amsmath, amssymb}
\usepackage{enumerate}
\usepackage{paralist}
\usepackage{fancyhdr}
\usepackage{titlesec}
\usepackage{verbatim}
\usepackage{/usr/share/R/share/texmf/tex/latex/Sweave}

\pagestyle{fancy}
\fancyhead[L]{STAT 579 HW4}
\fancyhead[R]{Xin Yin}
\titleformat{\section}{\em\Large} {$\dagger$ \thesection .}{10pt}{}

\begin{document}
    \section{}
    \begin{enumerate}[(a)]
    \item
    Let's start by reading in the time-series microarray expression values.
\begin{Schunk}
\begin{Sinput}
> TIMEPOINTS <- 11
> arab.array <- read.csv("http://www.public.iastate.edu/~maitra/stat579/datasets/diurnaldata.csv", 
+     header = T)
\end{Sinput}
\end{Schunk}
    \item
    Extract all the expression values by discarding the probe names, and
    convert this matrix into a 28810x11x2 array. Using \verb=apply()=
    to apply \verb=mean()= function on margins of 1st and 2nd
    dimensions, we can compute the mean values of two replicates for all probes and all timepoints. This will collapse the 3rd dimension and gives us a 28110x11 matrix.
\begin{Schunk}
\begin{Sinput}
> arab.expr.lvl <- as.matrix(arab.array[, 2:23])

# Convert the 22810x22 matrix to a 22810x11x2 array 
> arab.replicates <- array(arab.expr.lvl, c(dim(arab.expr.lvl)[1], 
+     TIMEPOINTS, 2))

# Calculate the mean abundance level across two replicates
> arab.means <- apply(arab.replicates, MARGIN = c(1, 2), FUN = mean)
\end{Sinput}
\end{Schunk}
    \item 
    \begin{inparaenum}
        \item[(d)]
        \item[(e)]
    \end{inparaenum}

    To standardize the matrix of mean expression values, we define a function \verb=standardize()=, which can be reused for standardizing the smaller matrix of mean expression levels later.
\begin{Schunk}
\begin{Sinput}
> standardize <- function(m) {
    # Standarize a matrix by substracting means for each row and divide the
    # data points by its standard deviation.
+     n.cols <- dim(m)[2]

    # Calculate mean of mean abundance level on each line (probe)
    # and replicate it N_timepoints times to span a matrix with the same
    # dimesion as m.
+     m.means <- matrix(rep(apply(m, MARGIN = 1, FUN = mean), times = n.cols), 
+         ncol = n.cols)
+     v.sds <- apply(m, MARGIN = 1, FUN = sd)

    # Standarize the original matrix m

    # For a matrix, the arithmatic division is performed columnwise. 
    # Since v.sds is a 28810x1 column vector, the following operatioin will
    # gurantee each probe's expression levels got divided by its own stddev.

+     return((m - m.means)/v.sds)
+ }
\end{Sinput}
\end{Schunk}
    Call function \verb=standardize()= to scale the mean expression values.
\begin{Schunk}
\begin{Sinput}
> arab.means.scaled <- standardize(arab.means)
\end{Sinput}
\end{Schunk}
    \item[(f)] Now we read in the second datafile, which stores the 20 mean expression values on 11 timepoints. Scale this matrix using function \verb=standardize= afterwards.
\begin{Schunk}
\begin{Sinput}
# Read in the 20 observation of expression levels
> array.means <- as.matrix(read.table("http://maitra.public.iastate.edu/stat579/datasets/micromeans.dat", 
+     header = F))
> array.means.scaled <- standardize(array.means)
\end{Sinput}
\end{Schunk}
    \item To calculate the Euclidean distance from each probe to any of these 20 means levels across 11 timepoints, we try to replicate both matrices into 3d arrays. As we superimpose these two arrays, we can compute Euclidean distance fairly easy using array manipulations.

    First we replicate the 28810x11 matrix along the 3rd dimension.
\begin{Schunk}
\begin{Sinput}
# Replicate scaled means for Arabidopsis data for 20 copys on the 3rd dimension
# Vectorize the matrix first, replicate, and map it to 3d space (array).

# So this is what the data will look like:

#  ^ z=20
#  |     _. x=28810
#  |     /|
#  |    /---/
#  |   /xxx/
#  |  /yyy/
#  | /zzz/
#  |/___/
#  +---------------> y = 11

#which will be replicated along the z-axis twenty fold
> arab.means.replicated <- array(arab.means.scaled, dim = c(dim(arab.means.scaled)[1], 
+     TIMEPOINTS, 20))
\end{Sinput}
\end{Schunk}
    
    Next we replicate the 20x11 matrix 28810 folds along the 1st dimension, such that given a fixed y on 2nd dimension, values along the 1st dimension are the same. 
\begin{Schunk}
\begin{Sinput}
# Now replicate the 20x11 matrix 28810 folds along x-axis.

#  ^ z=20
#  |     _. x=28810
#  |     /|
#  |----/---
#  |xxx/xxx|
#  |yy/yyyy|
#  |z/zzzzz|
#  |/wwwwww|
#  +---------------> y = 11

# We wish to replicate the 20x11 matrix in the 1st dimension such that, 
# within the first two dimensions, we see,

# Mx,1 Mx,2 Mx,3 ... Mx,11  \
# Mx,1 Mx,2 Mx,3 ... Mx,11   | 
#           ...              |
#           ...              |-- 28810 rows
#           ...              | 
# Mx,1 Mx,2 Mx,3 ... Mx,11   |
# Mx,1 Mx,2 Mx,3 ... Mx,11  /

# where x is the index of the 3rd dimension.

# in array.means.scaled, what we have is,

# M1,1 M1,2 M1,3 ... M1,11
# M2,1 M2,2 M2,3 ... M2,11
#         ...

# Remember that R does all matrice stuff (storing, vectorizing) in columnwise fashion,
# so transpose the matrix array.means.scaled (now 11x20), vectorize, and replicate 
# each element for 28810 times, will generate the data that looks like:

# M1,1 M1,1 ... (28810 times) M1,2 M1,2 .. (28810 times) ... (11 items) 
# M2,1 M2,1 ... (28810 times) ...

# Think about after vectorizing, you get a (11x20) elements row vector (v),
# [M1,1, M1,2 ... M1,20 M2,1 M2,2 ... M11, 1 ... M11, 20]

# Use rep(v, each=28810), you can think this process as growing a column out
# of every element in this row vector, using its value.

# So when you convert this replicated vector into a 3d array, 
# all the data fit into the correct spot.

# Consider fill these data points into a 28810x11x20 array, will look exactly
# like what we show above (28810x11 matrix), within the first two dimensions, 
# but 20 different matrices on different Z-values.

> array.means.replicated <- array(rep(as.vector(t(array.means.scaled)), 
+     each = dim(arab.means.scaled)[1]), dim = c(dim(arab.means.scaled)[1], 
+     TIMEPOINTS, 20))
\end{Sinput}
\end{Schunk}
    So now we have two arrays with the same dimensions. We superimpose them and do the math to calculate the Euclidean distances along the 2nd dimension, where the 28810x11 and 11x20 matrices are orthogonal.
    
\begin{Schunk}
\begin{Sinput}
# Note that the matrices spanned on z-y axis (20x11) and x-y axis (28810x11) 
# are orthogonal to each other. The overlapped segment contains 11 elements.

# Now we superimpose two 3d cube, compute the differences between two points 
# that overlap with each other, which gives a 28810x11x20 array.
# Compute the squared value for each point in this 3d space, and sum over y-axis. 
# This will end up with a 28810x20 matrix where each point indicates the Euclidean
# distances from the 11 timepoints in X-th probe to the 11 timepoints in Z-th row 
# of the 20x11 matrix.

> euclid.dist.comp <- (arab.means.replicated - array.means.replicated)^2
> euclid.dist <- apply(euclid.dist.comp, MARGIN = c(1, 3), FUN = sum)
\end{Sinput}
\end{Schunk}
    Sum up the squared differences between two values on the same coordinates across two arrays will collapse the 2nd dimension and gives us the squared Euclidean distances.

    Now we can find out, in the 28810x20 matrix, that which column has the minimal distance on each row, again, using the omnipotent \verb=apply()= function. Results are then tabulated.

\begin{Schunk}
\begin{Sinput}
# We don't necessarily need to compute the square root of each point, 
# since if Dx > Dy, Dx^2 > Dy^2 as well.
# Now we just find the Z that is closest to each given X using apply() function again.

# For each row in the 28810x10 matrix, we use function order() to sort the distances
# (squared) in increasing order, and the 1st element in sorted orders is the Z which 
# minimizes the Euclidean distance.

> closest.mean <- apply(euclid.dist, MARGIN = 1, function(x) order(x)[1])
> closest.tbl <- data.frame(table(closest.mean))
> names(closest.tbl) <- c("means", "freq")
> print(closest.tbl)
\end{Sinput}
\begin{Soutput}
   means freq
1      1 1043
2      2  959
3      3 1424
4      4 1641
5      5  449
6      6 1213
7      7 1221
8      8  838
9      9  807
10    10  891
11    11 1008
12    12  868
13    13 1033
14    14  709
15    15 3246
16    16 1361
17    17  823
18    18 1168
19    19 1109
20    20  999
\end{Soutput}
\end{Schunk}

    And are visualized as well, with a pie-chart.
\begin{Schunk}
\begin{Sinput}
> library(ggplot2)
> pie <- ggplot(closest.tbl, aes(x = factor(1), y = freq, fill = factor(means))) + 
+     geom_bar(width = 1)
> print(pie + coord_polar(theta = "y"))
\end{Sinput}
\end{Schunk}
\includegraphics{hw4-010}
    \end{enumerate}
\end{document}
